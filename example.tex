
\documentclass{lab_report}

\begin{document}

\course{Распределённые информационные системы}
\title{Многопроцессорные вычислительные системы и параллельные вычисления}
\labNumber{1}
\variant{11}
\author{Мазов В. И.}
\group{11М}
\teacher{Локтев Д. А.}

\maketitle

\tableofcontents

\section{Цель работы}
Основной целью лабораторной работы является...


\section{Задание}
\begin{task}[Улучшение агента ReflexAgent]
	Улучшите метод ReflexAgent...
\end{task}

\begin{task}[Улучшение алгоритма]
	Вам необходимо написать конкурентоспособного поискового агента...
\end{task}

\section{Выполнение работы}

\taskDoneSection
Используем метод потенциалов для оценки состояния. \smartref[Листинг]{lis:eval} содержит исходный код функции оценки, реализующей приведённый алгоритм.

% \begin{listing}{<language>}{<Caption>}{<label>} 
% "lis:" is added to label automatically
\begin{listing}{Python}{Функция оценки}{eval}
def evaluationFunction(self, currentGameState, action): 
	successorGameState = currentGameState.generatePacmanSuccessor(action)
	newPos = successorGameState.getPacmanPosition()
\end{listing}

% \smartref[<Caption>]{<label>}
\smartref[Рисунок]{fig:pacman} содержит результат запуска программы...

% \insertfigure{<filename>}{<Caption>}
% label "fig:<filename>" is set automatically
\insertfigure{pacman}{Работа программы}

\taskDoneSection
...

\section{Выводы}
В результате проделанной работы было ...

\end{document}
